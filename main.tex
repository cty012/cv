\documentclass[10pt, letterpaper]{article}

% Packages:
\usepackage[
    ignoreheadfoot, % set margins without considering header and footer
    top=2 cm, % seperation between body and page edge from the top
    bottom=2 cm, % seperation between body and page edge from the bottom
    left=2 cm, % seperation between body and page edge from the left
    right=2 cm, % seperation between body and page edge from the right
    footskip=1.0 cm, % seperation between body and footer
    % showframe % for debugging 
]{geometry} % for adjusting page geometry
\usepackage{titlesec} % for customizing section titles
\usepackage{tabularx} % for making tables with fixed width columns
\usepackage{array} % tabularx requires this
\usepackage[dvipsnames]{xcolor} % for coloring text
\definecolor{primaryColor}{RGB}{0, 0, 0} % define primary color
\usepackage{enumitem} % for customizing lists
\usepackage{fontawesome5} % for using icons
\usepackage{amsmath} % for math
\usepackage[
    pdftitle={Tianyue Cao's CV},
    pdfauthor={Tianyue Cao},
    pdfcreator={LaTeX with RenderCV},
    colorlinks=true,
    urlcolor=primaryColor
]{hyperref} % for links, metadata and bookmarks
\usepackage[pscoord]{eso-pic} % for floating text on the page
\usepackage{calc} % for calculating lengths
\usepackage{bookmark} % for bookmarks
\usepackage{lastpage} % for getting the total number of pages
\usepackage{changepage} % for one column entries (adjustwidth environment)
\usepackage{paracol} % for two and three column entries
\usepackage{ifthen} % for conditional statements
\usepackage{needspace} % for avoiding page brake right after the section title
\usepackage{iftex} % check if engine is pdflatex, xetex or luatex

% Ensure that generate pdf is machine readable/ATS parsable:
\input{glyphtounicode}
\pdfgentounicode=1
\usepackage[T1]{fontenc}
\usepackage[utf8]{inputenc}
\usepackage[default]{sourcesanspro}

% Some settings:
\raggedright
\AtBeginEnvironment{adjustwidth}{\partopsep0pt} % remove space before adjustwidth environment
\pagestyle{empty} % no header or footer
\setcounter{secnumdepth}{0} % no section numbering
\setlength{\parindent}{0pt} % no indentation
\setlength{\topskip}{0pt} % no top skip
\setlength{\columnsep}{0.15cm} % set column seperation
\pagenumbering{gobble} % no page numbering

\titleformat{\section}{\needspace{4\baselineskip}\bfseries\large}{}{0pt}{}[\vspace{1pt}\titlerule]

\titlespacing{\section}{
    % left space:
    -1pt
}{
    % top space:
    0.3 cm
}{
    % bottom space:
    0.2 cm
} % section title spacing

\renewcommand\labelitemi{$\vcenter{\hbox{\small$\bullet$}}$} % custom bullet points
\newenvironment{highlights}{
    \begin{itemize}[
        topsep=0.10 cm,
        parsep=0.10 cm,
        partopsep=0pt,
        itemsep=0pt,
        leftmargin=0 cm + 10pt
    ]
}{
    \end{itemize}
} % new environment for highlights

\newenvironment{highlightsforbulletentries}{
    \begin{itemize}[
        topsep=0.10 cm,
        parsep=0.10 cm,
        partopsep=0pt,
        itemsep=0pt,
        leftmargin=10pt
    ]
}{
    \end{itemize}
} % new environment for highlights for bullet entries

\newenvironment{onecolentry}{
    \begin{adjustwidth}{
        0 cm + 0.00001 cm
    }{
        0 cm + 0.00001 cm
    }
}{
    \end{adjustwidth}
} % new environment for one column entries

\newenvironment{twocolentry}[2][]{
    \onecolentry
    \def\secondColumn{#2}
    \setcolumnwidth{\fill, 4.5 cm}
    \begin{paracol}{2}
}{
    \switchcolumn \raggedleft \secondColumn
    \end{paracol}
    \endonecolentry
} % new environment for two column entries

\newenvironment{threecolentry}[3][]{
    \onecolentry
    \def\thirdColumn{#3}
    \setcolumnwidth{, \fill, 4.5 cm}
    \begin{paracol}{3}
    {\raggedright #2} \switchcolumn
}{
    \switchcolumn \raggedleft \thirdColumn
    \end{paracol}
    \endonecolentry
} % new environment for three column entries

\newenvironment{header}{
    \setlength{\topsep}{0pt}\par\kern\topsep\centering\linespread{1.5}
}{
    \par\kern\topsep
} % new environment for the header


\begin{document}
\newcommand{\AND}{\unskip
    \cleaders\copy\ANDbox\hskip\wd\ANDbox
    \ignorespaces
}
\newsavebox\ANDbox
\sbox\ANDbox{$|$}

\begin{header}
    \fontsize{25 pt}{25 pt}\selectfont Tianyue Cao

    \vspace{5 pt}

    \normalsize
    \mbox{\href{mailto:cao\_tianyue@hotmail.com}{cao\_tianyue@hotmail.com}}%
    \kern 5.0 pt%
    \AND%
    \kern 5.0 pt%
    \mbox{\href{tel:+86-156-0115-3047}{+86 156 0115 3047}}%
    \kern 5.0 pt%
    \AND%
    \kern 5.0 pt%
    \mbox{\href{tel:+1-609-250-3145}{+1 (609) 250-3145}}%
    \kern 5.0 pt%
    \AND%
    \kern 5.0 pt%
    \mbox{\href{https://github.com/cty012}{github.com/cty012}}%
    \kern 5.0 pt%
    \AND%
    \kern 5.0 pt%
    \mbox{1010 W University Ave, Urbana, IL 61801}%
\end{header}

\vspace{5 pt - 0.3 cm}


%% SECTION: Education
\section{Education}
    %% University of Illinois Urbana-Champaign
    \begin{twocolentry}{Aug 2021 \textendash\ May 2025}
        \textbf{University of Illinois Urbana-Champaign}, Urbana-Champaign, IL
    \end{twocolentry}

    \vspace{0.10 cm}
    \begin{onecolentry}
        \textit{B.S. in Mathematics \& Computer Science}
    \end{onecolentry}

    \vspace{0.10 cm}
    \begin{onecolentry}
    \begin{highlights}
        \item GPA: 3.89/4.0
        \item \textbf{Math coursework:} Linear algebra, Abstract algebra, Real analysis, Graph theory, Combinatorics, Euclidean/Non-Euclidean geometry, Ordinary differential equations
        \item \textbf{CS coursework:} Data structures, Algorithms, Numerical analysis, Formal models of computation
    \end{highlights}
    \end{onecolentry}


%% SECTION: Research Experience
\section{Research Experience}
    %%
    \begin{twocolentry}{Jan 2024 \textendash\ Nov 2024}
        \textbf{Chip games and multipartite graph paintability}
    \end{twocolentry}

    \vspace{0.10 cm}
    \begin{onecolentry}
        \href{https://arxiv.org/abs/2411.19462}{\textit{https://arxiv.org/abs/2411.19462}}
    \end{onecolentry}

    \vspace{0.10 cm}
    \begin{onecolentry}
    \begin{highlights}
        \item Investigated the problem of “online” multipartite graph coloring. Unlike the traditional graph coloring problem, the person coloring the multipartite graph gradually gains information about the graph as they color (applications include solving problems requiring graph coloring, but information is not given at once)
        \item Proved theorems that simplify the problem by reducing the effective number of cases by over 99\%
        \item Built a computational model that iterates through all possible moves to find the optimal strategy and developed a novel efficient algorithm to verify the correctness of the optimal strategy. This model not only confirms several currently known upper and lower bounds, but also inspired teammates to come up with new theorems
        \item Used computational model to find optimal strategies for small graphs, and made it into an interactable program
    \end{highlights}
    \end{onecolentry}

    \vspace{0.20 cm}

    %%
    \begin{twocolentry}{Sep 2019 \textendash\ May 2021}
        \textbf{Study of Sample Efficiency Improvements for Reinforcement Learning Algorithms}
    \end{twocolentry}

    \vspace{0.10 cm}
    \begin{onecolentry}
        \href{https://doi.org/10.1109/ISEC49744.2020.9397834}{\textit{10.1109/ISEC49744.2020.9397834}}
    \end{onecolentry}

    \vspace{0.10 cm}
    \begin{onecolentry}
    \begin{highlights}
        \item Designed two methods, importance sampling and dynamic epsilon, to improve the efficiency of deep Q-learning
        \item Conducted experiments to show that the dynamic epsilon method has a slight improvement on the efficiency
    \end{highlights}
    \end{onecolentry}


%% SECTION: Teaching Experience
\section{Teaching Experience}
    %% Course Assistant
    \begin{twocolentry}{Jan 2024 \textendash\ May 2024}
        \textbf{Course Assistant}, Urbana-Champaign, IL
    \end{twocolentry}

    \vspace{0.10 cm}
    \begin{onecolentry}
        \textit{UIUC CS 374: Introduction to Algorithms \& Models of Computation}
    \end{onecolentry}

    \vspace{0.10 cm}
    \begin{onecolentry}
    \begin{highlights}
        \item Graded students' homework and provided constructive feedback
        \item Answered student questions during discussion sections
        \item Hosted office hours and answered student questions about class materials or homework problems
    \end{highlights}
    \end{onecolentry}

    \vspace{0.20 cm}

    %% Course Assistant
    \begin{twocolentry}{Sep 2023 \textendash\ Dec 2023}
        \textbf{Course Assistant}, Urbana-Champaign, IL
    \end{twocolentry}

    \vspace{0.10 cm}
    \begin{onecolentry}
        \textit{UIUC CS 450: Numerical Analysis}
    \end{onecolentry}

    \vspace{0.10 cm}
    \begin{onecolentry}
    \begin{highlights}
        \item Graded students' homework and provided constructive feedback
        \item Answered questions on the forum
    \end{highlights}
    \end{onecolentry}


%% Awards
\section{Awards \& Honors}
    \begin{twocolentry}{Feb 2023, Mar 2022}
        University of Illinois Undergraduate Math Contest, \textit{1\textsuperscript{st}} place
    \end{twocolentry}

    \vspace{0.10 cm}
    \begin{twocolentry}{Aug 2021 \textendash\ May 2025}
        Dean's List (5 times)
    \end{twocolentry}

    \vspace{0.10 cm}
    \begin{twocolentry}{Aug 2020}
        IEEE Integrated STEM Education Conference (ISEC) Best Mathematical Research
    \end{twocolentry}

    \vspace{0.20 cm}
    \begin{twocolentry}{Dec 2019}
        The United States of America Computing Olympiad (USACO), Gold Division
    \end{twocolentry}

    \vspace{0.20 cm}
    \begin{twocolentry}{Jun 2019, Jun 2018}
        Mathematics Olympiad Program (MOP), Participant
    \end{twocolentry}
    
\section{Skills}
    \begin{onecolentry}
        \textbf{Programming Languages/Libraries:} Python, C/C++, HTML/CSS, JavaScript, Java, React, Node.js, SQL
    \end{onecolentry}

    \vspace{0.2 cm}

    \begin{onecolentry}
        \textbf{Natural Languages:} Chinese, English
    \end{onecolentry}

\end{document}
